%!TEX root = ../tikz-figures.tex

\section*{Crosshatch shapes}

\index{patterns library}
\index{library!patterns|see {patterns library}}
\index{overlap}
\index{arrows!hooked}

\begin{equation*}
	\begin{tikzpicture}
		\begin{scope}
			\draw[pattern=crosshatch] (0,0) circle (0.5);
			\draw[pattern=horizontal lines] (0,2) ellipse (1 and 0.75);
			\draw[pattern=north east lines] (0,4.5) ellipse (1.5 and 1);
			\draw[->] (0,0.55) -- (0,1.2);
			\draw[->] (0,2.8) -- (0,3.45);
		\end{scope}
		\begin{scope}[xshift=75]
			% \node at (0,2.5) {$\propto$};
		\end{scope}
		\begin{scope}[xshift=150]
			\draw[pattern=crosshatch] (0,0) circle (0.5);

			\draw[pattern=horizontal lines] (0,2) ellipse (1 and 0.75);
			\fill[white] (0,2) circle (0.5);
			\draw[pattern=crosshatch] (0,2) circle (0.5);

			\draw[pattern=north east lines] (0,4.5) ellipse (1.5 and 1);
			\fill[white] (0,4.5) ellipse (1 and 0.75);
			\draw[pattern=horizontal lines] (0,4.5) ellipse (1 and 0.75);
			\fill[white] (0,4.5) circle (0.5);
			\draw[pattern=crosshatch] (0,4.5) circle (0.5);
			\draw[right hook->] (0,0.55) -- (0,1.2);
			\draw[right hook->] (0,2.8) -- (0,3.45);
		\end{scope}
	\end{tikzpicture}
\end{equation*}

\begin{equation*}
	\index{transformation!rotate}
	\index{patterns library!coloured}
	\begin{tikzpicture}[xscale=0.8]
		\begin{scope}[rotate=270]
			\draw[pattern=crosshatch, pattern color=red] (0,0) circle (0.5);
			\draw[pattern=horizontal lines, pattern color=blue] (0,2) ellipse (1 and 0.75);
			\draw[pattern=north east lines, pattern color=olive] (0,4.5) ellipse (1.5 and 1);
			\draw[->] (0,0.55) -- (0,1.2);
			\draw[->] (0,2.8) -- (0,3.45);
		\end{scope}
		\begin{scope}[xshift=195]
			\draw (0,-1.5) -- (0,1.5);
		\end{scope}
		\begin{scope}[xshift=250, rotate=270]
			\draw[pattern=crosshatch, pattern color=red] (0,0) circle (0.5);

			\draw[pattern=horizontal lines, pattern color=blue] (0,2) ellipse (1 and 0.75);
			\fill[white] (0,2) circle (0.5);
			\draw[pattern=crosshatch, pattern color=red] (0,2) circle (0.5);

			\draw[pattern=north east lines, pattern color=olive] (0,4.5) ellipse (1.5 and 1);
			\fill[white] (0,4.5) ellipse (1 and 0.75);
			\draw[pattern=horizontal lines, pattern color=blue] (0,4.5) ellipse (1 and 0.75);
			\fill[white] (0,4.5) circle (0.5);
			\draw[pattern=crosshatch, pattern color=red] (0,4.5) circle (0.5);
			\draw[right hook->] (0,0.55) -- (0,1.2);
			\draw[right hook->] (0,2.8) -- (0,3.45);
		\end{scope}
	\end{tikzpicture}
\end{equation*}

\begin{equation*}
	\index{foreach}
	% \index{if statement!ifthenelse}
	\index{variable!pgfmathsetmacro}
	\index{legend}
	\index{patterns library}
	\index{clip}
	\index{path!partly rounded}
	\index{limit visualisation}
	\begin{tikzpicture}[xscale=1.3]
		\pgfmathsetmacro{\rightits}{4}; % Rightward iterations
		\pgfmathsetmacro{\leftits}{30}; % Leftward iterations
		\draw[rounded corners=10, pattern=dots] 
			({\rightits*2},0.5) -- (0,0.5) -- (0,-0.5) -- ({\rightits*2},-0.5);
		\begin{scope}
			\clip [rounded corners=10] ({\rightits*2},0.5) -- (0,0.5) -- (0,-0.5) -- ({\rightits*2},-0.5);
			\foreach \n in {1,...,\leftits}
			{
				\path[fill=white]
					({5/(3*\n+2)},0.5) rectangle ({5/(3*\n+1)},-0.5);
				\draw[pattern=crosshatch]
					({5/(3*\n+2)},0.5) rectangle ({5/(3*\n+1)},-0.5);
			}
			\draw[fill=black] (0,0.5) rectangle ({5/(3*\leftits)},-0.5);
		\end{scope}
		\pgfmathsetmacro{\rightitsminus}{\rightits-1}
		\foreach \n in {1,...,\rightitsminus} % Tikz at Maths doesn't like {\rightits-1}
		{
			\path[fill=white]
				({\n*2},0.5) 
					{[rounded corners=10] -- ({\n*2+1.414},0.5)
					-- ({\n*2+1.414},-0.5)}
					-- ({\n*2},-0.5)
					-- cycle;
			\draw[pattern=crosshatch]
				({\n*2},0.5) 
					{[rounded corners=10] -- ({\n*2+1.414},0.5)
					-- ({\n*2+1.414},-0.5)}
					-- ({\n*2},-0.5)
					-- cycle;
			\draw ({\n*2+1.414},0.5) -- ({\n*2+1.414},0.6);
			\node at ({\n*2+1.414},0.85) {$\n+\nicefrac 1 {\sqrt 2}$};
			\draw ({\n*2},-0.5) -- ({\n*2},-0.6);
			\node at ({\n*2},-0.85) {$\n$};
		}
		\node at ({\rightits*2 + 0.5},0) {$\cdots$};
		\begin{scope}[xshift=75,yshift=-50]
			\node[draw, pattern=dots, outer sep=0pt,inner sep=0,minimum size=15] [label=right:{$\aleph_0$-dense}] {};
			\node[draw, pattern=crosshatch, outer sep=0pt,inner sep=0,minimum size=15] [label=right:{$\aleph_1$-dense}] at (3,0) {};
		\end{scope}
	\end{tikzpicture}
\end{equation*}


\begin{equation*}
	\index{foreach}
	\index{if statement!ifthenelse}
	\index{variable!pgfmathsetmacro}
	\index{array}
	\index{array!comparing values}
	\index{patterns library}
	\index{clip}
	\index{path!partly rounded}
	\index{limit visualisation}
	\begin{tikzpicture}[xscale=1.3]
		\def\subset{{1,0,0,1,0,0,1,1,1}}
		\pgfmathsetmacro{\subsetmax}{8}
		\draw[rounded corners=10] 
			(8,0.5) -- (0,0.5) -- (0,-0.5) -- (8,-0.5);
		\begin{scope}
			\clip [rounded corners=10] (8,0.5) -- (0,0.5) -- (0,-0.5) -- (8,-0.5);
			\fill[pattern=dots] (1,0.5) rectangle (8,-0.5);
			\foreach \n in {0,...,\subsetmax}
			{
				\pgfmathsetmacro{\xmin}{{9*(1-atan((2*\n+1)/7)/90)-1}}
				\pgfmathsetmacro{\xmax}{{9*(1-atan((2*\n+2)/7)/90)-1}}
				\pgfmathsetmacro{\xmed}{{(\xmin+\xmax)/2}}
				\pgfmathsetmacro\subsetvalue{\subset[\n]}
				\ifthenelse{\equal{\subsetvalue}{1}}
				{
					\path[fill=white]
						(\xmin,0.5) 
							{[rounded corners=5] -- (\xmax,0.5)
							-- (\xmax,-0.5)}
							-- (\xmin,-0.5)
							-- cycle;
					\draw[fill=black]
						(\xmin,0.5) 
							{[rounded corners=5] -- (\xmax,0.5)
							-- (\xmax,-0.5)}
							-- (\xmin,-0.5)
							-- cycle;
				}
				{
					\path[fill=white, rounded corners=5]
						(\xmin,0.5) 
							-- (\xmax,0.5)
							-- (\xmax,-0.5)
							-- (\xmin,-0.5)
							-- cycle;
					\draw[fill=black, rounded corners=5]
						(\xmin,0.5) 
							-- (\xmax,0.5)
							-- (\xmax,-0.5)
							-- (\xmin,-0.5)
							-- cycle;
				}
				\coordinate (i\n) at (\xmed,-0.5);
			}
			\node at (0.6,0) {$\cdots$};
		\end{scope}
		\begin{scope}[yshift=-70]
			\node at (0.5,0) {$A:$};
			\node (a0) at (6.5,0) {$0$};
			\node (a1) at (6.0,0) {$1$};
			\node (a2) at (5.5,0) {$2$};
			\node (a3) at (5.0,0) {$3$};
			\node (a4) at (4.5,0) {$4$};
			\node (a5) at (4.0,0) {$5$};
			\node (a6) at (3.5,0) {$6$};
			\node (a7) at (3.0,0) {$7$};
			\node (a8) at (2.5,0) {$8$};
			\node at (1,0) {$\bigg[$};
			\node at (7,0) {$\bigg]$};
			\node at (1.75,0) {$\cdots$};
			\begin{scope}[every node/.style={draw, circle, minimum size=15pt}]
				\node (a0c) at (a0) {};
				\node[white] (a1c) at (a1) {};
				\node[white] (a2c) at (a2) {};
				\node (a3c) at (a3) {};
				\node[white] (a4c) at (a4) {};
				\node[white] (a5c) at (a5) {};
				\node (a6c) at (a6) {};
				\node (a7c) at (a7) {};
				\node (a8c) at (a8) {};
			\end{scope}
			\foreach \n in {0,...,\subsetmax}
			{
				\draw[->] (a\n c) -- ($(i\n)-(0,0.1)$);
			}
			\node at (4,-1) {$A = \{\ldots, 8, 7 ,6, 3, 0\}$};
		\end{scope}
	\end{tikzpicture}
\end{equation*}

\begin{equation*}
	\index{foreach}
	\index{variable!pgfmathsetmacro}
	\index{patterns library}
	\index{clip}
	\index{path!partly rounded}
	\index{limit visualisation}
	\begin{tikzpicture}[xscale=1.3]
		\begin{scope}
			\node at (-0.5,0) {(i)};
			\pgfmathsetmacro{\intervalmax}{25}
			\draw[rounded corners=10] 
				(8,0.5) -- (0,0.5) -- (0,-0.5) -- (8,-0.5);
			\begin{scope}
				\clip [rounded corners=10] (8,0.5) -- (0,0.5) -- (0,-0.5) -- (8,-0.5);
				\fill[pattern=dots] (0,0.5) rectangle (8,-0.5);
				\foreach \n in {0,...,\intervalmax}
				{
					\pgfmathsetmacro{\xmin}{{9*(1-atan((2*\n+1)/7)/90)-0.8}}
					\pgfmathsetmacro{\xmax}{{9*(1-atan((2*\n+2)/7)/90)-0.8}}
					\pgfmathsetmacro{\rounding}{{10*(1-atan((2*\n+2)/7)/90)}}
					% \pgfmathsetmacro{\rounding}{{5}}
					\path[fill=white]
						(\xmin,0.5) 
							{[rounded corners=\rounding] -- (\xmax,0.5)
							-- (\xmax,-0.5)}
							-- (\xmin,-0.5)
							-- cycle;
					\draw[fill=black]
						(\xmin,0.5) 
							{[rounded corners=\rounding] -- (\xmax,0.5)
							-- (\xmax,-0.5)}
							-- (\xmin,-0.5)
							-- cycle;
				}
			\end{scope}
		\end{scope}
		\begin{scope}[yshift=-50]
			\node at (-0.5,0) {(ii)};
			\pgfmathsetmacro{\intervalamax}{25}
			\pgfmathsetmacro{\intervalamin}{4}
			\pgfmathsetmacro{\intervalbmax}{25}
			\pgfmathsetmacro{\intervalbmin}{2}
			\draw[rounded corners=10] 
				(8,0.5) -- (0,0.5) -- (0,-0.5) -- (8,-0.5);
			\begin{scope}
				\clip [rounded corners=10] (8,0.5) -- (0,0.5) -- (0,-0.5) -- (8,-0.5);
				\fill[pattern=dots] (0,0.5) rectangle (8,-0.5);
				\foreach \n in {\intervalamin,...,\intervalamax}
				{
					\pgfmathsetmacro{\xmin}{{9*(1-atan((2*\n+1)/7)/90)-0.8}}
					\pgfmathsetmacro{\xmax}{{9*(1-atan((2*\n+2)/7)/90)-0.8}}
					\pgfmathsetmacro{\rounding}{{10*(1-atan((2*\n+2)/7)/90)}}
					% \pgfmathsetmacro{\rounding}{{5}}
					\path[fill=white]
						(\xmin,0.5) 
							{[rounded corners=\rounding] -- (\xmax,0.5)
							-- (\xmax,-0.5)}
							-- (\xmin,-0.5)
							-- cycle;
					\draw[fill=black]
						(\xmin,0.5) 
							{[rounded corners=\rounding] -- (\xmax,0.5)
							-- (\xmax,-0.5)}
							-- (\xmin,-0.5)
							-- cycle;
				}
				\foreach \n in {\intervalbmin,...,\intervalbmax}
				{
					\pgfmathsetmacro{\xmin}{{9*(1-atan((2*\n+1)/7)/90)-1+3.4}}
					\pgfmathsetmacro{\xmax}{{9*(1-atan((2*\n+2)/7)/90)-1+3.4}}
					\pgfmathsetmacro{\rounding}{{10*(1-atan((2*\n+2)/7)/90)}}
					% \pgfmathsetmacro{\rounding}{{5}}
					\path[fill=white]
						(\xmin,0.5) 
							{[rounded corners=\rounding] -- (\xmax,0.5)
							-- (\xmax,-0.5)}
							-- (\xmin,-0.5)
							-- cycle;
					\draw[fill=black]
						(\xmin,0.5) 
							{[rounded corners=\rounding] -- (\xmax,0.5)
							-- (\xmax,-0.5)}
							-- (\xmin,-0.5)
							-- cycle;
				}
			\end{scope}
		\end{scope}
		\begin{scope}[yshift=-100]
			\node at (-0.5,0) {(iii)};
			\pgfmathsetmacro{\intervalmax}{6}
			\draw[rounded corners=10] 
				(8,0.5) -- (0,0.5) -- (0,-0.5) -- (8,-0.5);
			\begin{scope}
				\clip [rounded corners=10] (8,0.5) -- (0,0.5) -- (0,-0.5) -- (8,-0.5);
				\fill[pattern=dots] (0,0.5) rectangle (8,-0.5);
				\foreach \n in {0,...,\intervalmax}
				{
					\pgfmathsetmacro{\xmin}{{9*(1-atan((2*\n+1)/7)/90)-1-2.1}}
					\pgfmathsetmacro{\xmax}{{9*(1-atan((2*\n+2)/7)/90)-1-2.1}}
					\pgfmathsetmacro{\rounding}{{10*(1-atan((2*\n+2)/7)/90)}}
					% \pgfmathsetmacro{\rounding}{{5}}
					\path[fill=white]
						(\xmin,0.5) 
							{[rounded corners=\rounding] -- (\xmax,0.5)
							-- (\xmax,-0.5)}
							-- (\xmin,-0.5)
							-- cycle;
					\draw[fill=black]
						(\xmin,0.5) 
							{[rounded corners=\rounding] -- (\xmax,0.5)
							-- (\xmax,-0.5)}
							-- (\xmin,-0.5)
							-- cycle;
				}
			\end{scope}
		\end{scope}
	\end{tikzpicture}
\end{equation*}

\begin{equation*}
	\index{foreach!change variable}
	\index{variable!global}
	\index{arrows!mid-line}
	\index{clip}
	\index{path!partly rounded}
	\index{variable!pgfmathsetmacro}
	\index{array!two-dimensional}
	\index{decorations library}
	\begin{tikzpicture}
		\begin{scope}
			\begin{scope}
				\clip[rounded corners=3] (10,0) -- (0,0) -- (0,0.25) -- (10,0.25);
				\def\cols{{{"black",0.05},{"green",0.03},{"black",0.04},{"blue",0.02},{"red",0.08},{"black",0.05},{"white",0.05},{"blue",0.1},{"white",0.09},{"green",0.1},{"red",0.1},{"white",0.05},{"blue",0.25},{"red",0.3},{"black",0.1},{"green",0.4},{"red",0.35},{"green",0.5},{"white",1},{"blue",0.75},{"black",0.2},{"red",1},{"blue",0.5},{"green",1.45},{"red",0.5},{"green",2}}}
				\pgfmathsetmacro{\colmax}{25}
				\pgfmathsetmacro{\xpos}{0}
				\foreach \i in {0,...,\colmax}
				{
					\pgfmathsetmacro{\nextxpos}{\xpos+\cols[\i][1]}
					\pgfmathsetmacro{\col}{\cols[\i][0]}
					\fill[fill=\col] (\xpos,0) rectangle (\nextxpos,0.25);
					\global\let\xpos=\nextxpos
				}
			\end{scope}
			\draw[rounded corners=3] (10,0) -- (0,0) -- (0,0.25) -- (10,0.25);
			\node at (-0.3,0.125) {$c$};
			\draw (3.25,0) -- (3.25,0.25) node[above] (x) {$x$};
			\node[right = 10pt of x, yshift=1pt] {$c(x) = \text{white}$};
		\end{scope}
		\begin{scope}[yscale=0.6,yshift=-35,xscale=0.5]
			\node at (-1,0) {$p$};
			\draw[->-]  (0,1) --  (0.0,-1);
			\draw[->-]  (1,1) --  (0.5,-1);
			\draw[->-]  (2,1) --  (0.8,-1);
			\draw[->-]  (3,1) --  (2.0,-1);
			\draw[->-]  (4,1) --  (4.2,-1);
			\draw[->-]  (5,1) --  (4.4,-1);
			\draw[->-]  (6,1) --  (4.5,-1);
			\draw[->-]  (7,1) --  (6.8,-1);
			\draw[->-]  (8,1) --  (8.0,-1);
			\draw[->-]  (9,1) --  (9.1,-1);
			\draw[->-] (10,1) -- (10.2,-1);
			\draw[->-] (11,1) -- (12.0,-1);
			\draw[->-] (12,1) -- (12.1,-1);
			\draw[->-] (13,1) -- (12.3,-1);
			\draw[->-] (14,1) -- (12.9,-1);
			\draw[->-] (15,1) -- (16.0,-1);
			\draw[->-] (16,1) -- (17.0,-1);
			\draw[->-] (17,1) -- (18.0,-1);
			\draw[->-] (18,1) -- (19.0,-1);
			\draw[->-] (19,1) -- (19.5,-1);
			\draw[->-] (20,1) -- (20.0,-1);
		\end{scope}
		\begin{scope}[yshift=-50]
			\begin{scope}
				\clip[rounded corners=3] (10,0) -- (0,0) -- (0,0.25) -- (10,0.25);
				\def\cols{{{"blue",0.04},{"black",0.05},{"green",0.03},{"black",0.02},{"red",0.09},{"blue",0.08},{"white",0.05},{"black",0.05},{"red",0.1},{"white",0.1},{"green",0.25},{"white",0.11},{"black",0.1},{"blue",0.05},{"red",0.1},{"blue",0.6},{"green",0.75},{"red",0.4},{"green",0.35},{"red",0.5},{"white",0.2},{"black",1.45},{"red",1},{"blue",0.5},{"black",1.2},{"green",3},{"blue",0.8},{"white",0.9},{"black",1.1},{"red",4}}}
				\pgfmathsetmacro{\colmax}{29}
				\pgfmathsetmacro{\xpos}{0}
				\foreach \i in {0,...,\colmax}
				{
					\pgfmathsetmacro{\nextxpos}{\xpos+\cols[\i][1]}
					\pgfmathsetmacro{\col}{\cols[\i][0]}
					\fill[fill=\col] (\xpos,0) rectangle (\nextxpos,0.25);
					\global\let\xpos=\nextxpos
				}
			\end{scope}
			\draw[rounded corners=3] (10,0) -- (0,0) -- (0,0.25) -- (10,0.25);
			\node at (-0.3,0.125) {$d$};
			\draw (2.3,0.25) -- (2.3,0) node[below] (px) {$p(x)$};
			\node[right = 10pt of px] {$d(p(x)) = \text{green}$};
		\end{scope}
	\end{tikzpicture}
\end{equation*}