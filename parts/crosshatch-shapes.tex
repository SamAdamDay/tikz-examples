%!TEX root = ../tikz-figures.tex

\section*{Crosshatch shapes}

\index{patterns library}
\index{library!patterns|see {patterns library}}
\index{overlap}
\index{arrows!hooked}

\begin{equation*}
	\begin{tikzpicture}
		\begin{scope}
			\draw[pattern=crosshatch] (0,0) circle (0.5);
			\draw[pattern=horizontal lines] (0,2) ellipse (1 and 0.75);
			\draw[pattern=north east lines] (0,4.5) ellipse (1.5 and 1);
			\draw[->] (0,0.55) -- (0,1.2);
			\draw[->] (0,2.8) -- (0,3.45);
		\end{scope}
		\begin{scope}[xshift=75]
			% \node at (0,2.5) {$\propto$};
		\end{scope}
		\begin{scope}[xshift=150]
			\draw[pattern=crosshatch] (0,0) circle (0.5);

			\draw[pattern=horizontal lines] (0,2) ellipse (1 and 0.75);
			\fill[white] (0,2) circle (0.5);
			\draw[pattern=crosshatch] (0,2) circle (0.5);

			\draw[pattern=north east lines] (0,4.5) ellipse (1.5 and 1);
			\fill[white] (0,4.5) ellipse (1 and 0.75);
			\draw[pattern=horizontal lines] (0,4.5) ellipse (1 and 0.75);
			\fill[white] (0,4.5) circle (0.5);
			\draw[pattern=crosshatch] (0,4.5) circle (0.5);
			\draw[right hook->] (0,0.55) -- (0,1.2);
			\draw[right hook->] (0,2.8) -- (0,3.45);
		\end{scope}
	\end{tikzpicture}
\end{equation*}

\begin{equation*}
	\index{transformation!rotate}
	\index{patterns library!coloured}
	\begin{tikzpicture}[xscale=0.8]
		\begin{scope}[rotate=270]
			\draw[pattern=crosshatch, pattern color=red] (0,0) circle (0.5);
			\draw[pattern=horizontal lines, pattern color=blue] (0,2) ellipse (1 and 0.75);
			\draw[pattern=north east lines, pattern color=olive] (0,4.5) ellipse (1.5 and 1);
			\draw[->] (0,0.55) -- (0,1.2);
			\draw[->] (0,2.8) -- (0,3.45);
		\end{scope}
		\begin{scope}[xshift=195]
			\draw (0,-1.5) -- (0,1.5);
		\end{scope}
		\begin{scope}[xshift=250, rotate=270]
			\draw[pattern=crosshatch, pattern color=red] (0,0) circle (0.5);

			\draw[pattern=horizontal lines, pattern color=blue] (0,2) ellipse (1 and 0.75);
			\fill[white] (0,2) circle (0.5);
			\draw[pattern=crosshatch, pattern color=red] (0,2) circle (0.5);

			\draw[pattern=north east lines, pattern color=olive] (0,4.5) ellipse (1.5 and 1);
			\fill[white] (0,4.5) ellipse (1 and 0.75);
			\draw[pattern=horizontal lines, pattern color=blue] (0,4.5) ellipse (1 and 0.75);
			\fill[white] (0,4.5) circle (0.5);
			\draw[pattern=crosshatch, pattern color=red] (0,4.5) circle (0.5);
			\draw[right hook->] (0,0.55) -- (0,1.2);
			\draw[right hook->] (0,2.8) -- (0,3.45);
		\end{scope}
	\end{tikzpicture}
\end{equation*}

\begin{equation*}
	\index{foreach}
	% \index{if statement!ifthenelse}
	\index{variable!pgfmathsetmacro}
	\index{legend}
	\index{patterns library}
	\index{clip}
	\index{path!partly rounded}
	\index{limit visualisation}
	\begin{tikzpicture}[xscale=1.3]
		\pgfmathsetmacro{\rightits}{4}; % Rightward iterations
		\pgfmathsetmacro{\leftits}{30}; % Leftward iterations
		\draw[rounded corners=10, pattern=dots] 
			({\rightits*2},0.5) -- (0,0.5) -- (0,-0.5) -- ({\rightits*2},-0.5);
		\begin{scope}
			\clip [rounded corners=10] ({\rightits*2},0.5) -- (0,0.5) -- (0,-0.5) -- ({\rightits*2},-0.5);
			\foreach \n in {1,...,\leftits}
			{
				\path[fill=white]
					({5/(3*\n+2)},0.5) rectangle ({5/(3*\n+1)},-0.5);
				\draw[pattern=crosshatch]
					({5/(3*\n+2)},0.5) rectangle ({5/(3*\n+1)},-0.5);
			}
			\draw[fill=black] (0,0.5) rectangle ({5/(3*\leftits)},-0.5);
		\end{scope}
		\pgfmathsetmacro{\rightitsminus}{\rightits-1}
		\foreach \n in {1,...,\rightitsminus} % Tikz at Maths doesn't like {\rightits-1}
		{
			\path[fill=white]
				({\n*2},0.5) 
					{[rounded corners=10] -- ({\n*2+1.414},0.5)
					-- ({\n*2+1.414},-0.5)}
					-- ({\n*2},-0.5)
					-- cycle;
			\draw[pattern=crosshatch]
				({\n*2},0.5) 
					{[rounded corners=10] -- ({\n*2+1.414},0.5)
					-- ({\n*2+1.414},-0.5)}
					-- ({\n*2},-0.5)
					-- cycle;
			\draw ({\n*2+1.414},0.5) -- ({\n*2+1.414},0.6);
			\node at ({\n*2+1.414},0.85) {$\n+\nicefrac 1 {\sqrt 2}$};
			\draw ({\n*2},-0.5) -- ({\n*2},-0.6);
			\node at ({\n*2},-0.85) {$\n$};
		}
		\node at ({\rightits*2 + 0.5},0) {$\cdots$};
		\begin{scope}[xshift=75,yshift=-50]
			\node[draw, pattern=dots, outer sep=0pt,inner sep=0,minimum size=15] [label=right:{$\aleph_0$-dense}] {};
			\node[draw, pattern=crosshatch, outer sep=0pt,inner sep=0,minimum size=15] [label=right:{$\aleph_1$-dense}] at (3,0) {};
		\end{scope}
	\end{tikzpicture}
\end{equation*}


\begin{equation*}
	\index{foreach}
	\index{if statement!ifthenelse}
	\index{variable!pgfmathsetmacro}
	\index{array}
	\index{array!comparing values}
	\index{patterns library}
	\index{clip}
	\index{path!partly rounded}
	\index{limit visualisation}
	\begin{tikzpicture}[xscale=1.3]
		\def\subset{{1,0,0,1,0,0,1,1,1}}
		\pgfmathsetmacro{\subsetmax}{8}
		\draw[rounded corners=10] 
			(8,0.5) -- (0,0.5) -- (0,-0.5) -- (8,-0.5);
		\begin{scope}
			\clip [rounded corners=10] (8,0.5) -- (0,0.5) -- (0,-0.5) -- (8,-0.5);
			\fill[pattern=dots] (1,0.5) rectangle (8,-0.5);
			\foreach \n in {0,...,\subsetmax}
			{
				\pgfmathsetmacro{\xmin}{{9*(1-atan((2*\n+1)/7)/90)-1}}
				\pgfmathsetmacro{\xmax}{{9*(1-atan((2*\n+2)/7)/90)-1}}
				\pgfmathsetmacro\subsetvalue{\subset[\n]}
				\ifthenelse{\equal{\subsetvalue}{1}}
				{
					\draw[fill=black]
						(\xmin,0.5) 
							{[rounded corners=5] -- (\xmax,0.5)
							-- (\xmax,-0.5)}
							-- (\xmin,-0.5)
							-- cycle;
				}
				{
					\draw[fill=black, rounded corners=5]
						(\xmin,0.5) 
							-- (\xmax,0.5)
							-- (\xmax,-0.5)
							-- (\xmin,-0.5)
							-- cycle;
				}
				\node at (0.6,0) {$\cdots$};
			}
		\end{scope}
		\node at (4,-1) {$A=\{\subset,\ldots\}$};
	\end{tikzpicture}
\end{equation*}

\begin{equation*}
	\index{foreach}
	\index{variable!pgfmathsetmacro}
	\index{patterns library}
	\index{clip}
	\index{path!partly rounded}
	\index{limit visualisation}
	\begin{tikzpicture}[xscale=1.3]
		\begin{scope}
			\node at (-0.5,0) {(i)};
			\pgfmathsetmacro{\intervalmax}{25}
			\draw[rounded corners=10] 
				(8,0.5) -- (0,0.5) -- (0,-0.5) -- (8,-0.5);
			\begin{scope}
				\clip [rounded corners=10] (8,0.5) -- (0,0.5) -- (0,-0.5) -- (8,-0.5);
				\fill[pattern=dots] (0,0.5) rectangle (8,-0.5);
				\foreach \n in {0,...,\intervalmax}
				{
					\pgfmathsetmacro{\xmin}{{9*(1-atan((2*\n+1)/7)/90)-0.8}}
					\pgfmathsetmacro{\xmax}{{9*(1-atan((2*\n+2)/7)/90)-0.8}}
					\pgfmathsetmacro{\rounding}{{10*(1-atan((2*\n+2)/7)/90)}}
					% \pgfmathsetmacro{\rounding}{{5}}
					\path[fill=white]
						(\xmin,0.5) 
							{[rounded corners=\rounding] -- (\xmax,0.5)
							-- (\xmax,-0.5)}
							-- (\xmin,-0.5)
							-- cycle;
					\draw[fill=black]
						(\xmin,0.5) 
							{[rounded corners=\rounding] -- (\xmax,0.5)
							-- (\xmax,-0.5)}
							-- (\xmin,-0.5)
							-- cycle;
				}
			\end{scope}
		\end{scope}
		\begin{scope}[yshift=-50]
			\node at (-0.5,0) {(ii)};
			\pgfmathsetmacro{\intervalamax}{25}
			\pgfmathsetmacro{\intervalamin}{4}
			\pgfmathsetmacro{\intervalbmax}{25}
			\pgfmathsetmacro{\intervalbmin}{2}
			\draw[rounded corners=10] 
				(8,0.5) -- (0,0.5) -- (0,-0.5) -- (8,-0.5);
			\begin{scope}
				\clip [rounded corners=10] (8,0.5) -- (0,0.5) -- (0,-0.5) -- (8,-0.5);
				\fill[pattern=dots] (0,0.5) rectangle (8,-0.5);
				\foreach \n in {\intervalamin,...,\intervalamax}
				{
					\pgfmathsetmacro{\xmin}{{9*(1-atan((2*\n+1)/7)/90)-0.8}}
					\pgfmathsetmacro{\xmax}{{9*(1-atan((2*\n+2)/7)/90)-0.8}}
					\pgfmathsetmacro{\rounding}{{10*(1-atan((2*\n+2)/7)/90)}}
					% \pgfmathsetmacro{\rounding}{{5}}
					\path[fill=white]
						(\xmin,0.5) 
							{[rounded corners=\rounding] -- (\xmax,0.5)
							-- (\xmax,-0.5)}
							-- (\xmin,-0.5)
							-- cycle;
					\draw[fill=black]
						(\xmin,0.5) 
							{[rounded corners=\rounding] -- (\xmax,0.5)
							-- (\xmax,-0.5)}
							-- (\xmin,-0.5)
							-- cycle;
				}
				\foreach \n in {\intervalbmin,...,\intervalbmax}
				{
					\pgfmathsetmacro{\xmin}{{9*(1-atan((2*\n+1)/7)/90)-1+3.4}}
					\pgfmathsetmacro{\xmax}{{9*(1-atan((2*\n+2)/7)/90)-1+3.4}}
					\pgfmathsetmacro{\rounding}{{10*(1-atan((2*\n+2)/7)/90)}}
					% \pgfmathsetmacro{\rounding}{{5}}
					\path[fill=white]
						(\xmin,0.5) 
							{[rounded corners=\rounding] -- (\xmax,0.5)
							-- (\xmax,-0.5)}
							-- (\xmin,-0.5)
							-- cycle;
					\draw[fill=black]
						(\xmin,0.5) 
							{[rounded corners=\rounding] -- (\xmax,0.5)
							-- (\xmax,-0.5)}
							-- (\xmin,-0.5)
							-- cycle;
				}
			\end{scope}
		\end{scope}
		\begin{scope}[yshift=-100]
			\node at (-0.5,0) {(iii)};
			\pgfmathsetmacro{\intervalmax}{6}
			\draw[rounded corners=10] 
				(8,0.5) -- (0,0.5) -- (0,-0.5) -- (8,-0.5);
			\begin{scope}
				\clip [rounded corners=10] (8,0.5) -- (0,0.5) -- (0,-0.5) -- (8,-0.5);
				\fill[pattern=dots] (0,0.5) rectangle (8,-0.5);
				\foreach \n in {0,...,\intervalmax}
				{
					\pgfmathsetmacro{\xmin}{{9*(1-atan((2*\n+1)/7)/90)-1-2.1}}
					\pgfmathsetmacro{\xmax}{{9*(1-atan((2*\n+2)/7)/90)-1-2.1}}
					\pgfmathsetmacro{\rounding}{{10*(1-atan((2*\n+2)/7)/90)}}
					% \pgfmathsetmacro{\rounding}{{5}}
					\path[fill=white]
						(\xmin,0.5) 
							{[rounded corners=\rounding] -- (\xmax,0.5)
							-- (\xmax,-0.5)}
							-- (\xmin,-0.5)
							-- cycle;
					\draw[fill=black]
						(\xmin,0.5) 
							{[rounded corners=\rounding] -- (\xmax,0.5)
							-- (\xmax,-0.5)}
							-- (\xmin,-0.5)
							-- cycle;
				}
			\end{scope}
		\end{scope}
	\end{tikzpicture}
\end{equation*}