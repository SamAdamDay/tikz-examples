%!TEX root = ../tikz-figures.tex

\section*{Trees}

\begin{equation*}
	\index{foreach}
	\index{foreach!backwards}
	\index{variable!pgfmathsetmacro}
	\index{variable!from expression}
	\index{variable!pgfmathparse}
	\index{clip}
	\index{path!curved}
	\index{node on path!fraction position}
	\index{math functions!mod}
	\begin{tikzpicture}
		\pgfmathsetmacro{\qpoints}{4}; % Number of q_n's to draw
		\pgfmathsetmacro{\qlabels}{2}; % Number of q_n's to label
		\begin{scope}
			\coordinate (x) at (0,0);
			\clip ($(x)+(135:7)$) rectangle ($(x)+(5,0)$);
			\fill[lightgray] (x) -- ($(x)+(135:3)$) .. controls +(1,1) and +(-1,-0.5) .. ($(x)+(45:3)$) -- (x);
			\draw (x) edge +(135:7) edge +(45:7);
			\draw ($(x)+(135:3)$) .. controls +(1,1) and +(-1,-0.5) .. ($(x)+(45:3)$)
				coordinate[pos=0.2] (q)
				coordinate[pos=0.8] (qinf);
			\draw (x) edge (q) edge (qinf);
			\fill[lightgray!50!white] (q) -- +(110:4) -- +(70:4) --cycle;
			\node[point] at (q) [label=below left:$q$] {};
			\draw (q) edge +(110:4) edge +(70:4);
			\draw[dotted] (qinf) edge +(110:4) edge +(70:4);
			\foreach \n in {0, ...,\qpoints}
			{
				\coordinate (q\n) at ($(qinf) + (0,{2/(2^\n)})$) {};
			}
			\foreach \n in {0, ...,\qpoints}
			{
				\pgfmathsetmacro{\m}{\qpoints-\n}
				\pgfmathparse{100/(mod(\m,2)+1)}
				\fill[lightgray!\pgfmathresult!white] (q\m) -- +(110:4) -- +(70:4) --cycle;
				\draw (q\m) edge +(110:4) edge +(70:4);
				\node[point] at (q\m) {};
			}
			% \draw (qinf) -- (q0);
			\draw node at ($(x) + (3,1)$) {$\us x^{\mathrm{NonPara}(X)}$} edge[->] ($(x) + (0,1.5)$);
		\end{scope}
		\foreach \n in {0,...,\qlabels}
		{
			\draw node at ($(qinf) + (0,{2/(2^\n)}) + (1.5,1)$) {$q_\n$} edge[->] (q\n);
		}
		\node[point] at (x) [label=right:$x$] {};
	\end{tikzpicture}
\end{equation*}

\begin{equation*}
	\index{path!curved}
	\index{node on path}
	\index{function plot}
	\index{transformation!slant}
	\begin{tikzpicture}
		\begin{scope}
			\filldraw[lightgray] (0,0) node[point] {} -- (-2,4) -- (0,4) --cycle; 
			\draw[dotted] (0,0) -- (-2,4);
			\draw[dotted] (0,0) -- (2,4);
			\draw (0,3) node (Ca) {} -- (1.5,4) node (Cb) {};
			\draw[decoration={calligraphic brace,amplitude=5pt}, decorate, line width=1.25pt]
				($(Cb) + (0.1,-0.1)$) -- node[below right] {$C$} ($(Ca) + (0.1,-0.1)$);
			\node at (-0.75,3) {$\dom(\ell)$};
		\end{scope}
		\begin{scope}[xshift=-120, yshift=-110, scale=1.5]
			\draw[->] (0,0) -- node[left] {$f$} (0,2);
			\draw (0,0) node (a1a) {} -- (2,0) node (a1b) {};
			\draw[decoration={calligraphic brace,amplitude=5pt}, decorate, line width=1.25pt]
				($(a1b) + (0,-0.1)$) -- node[yshift=-12] {$C$} ($(a1a) + (0,-0.1)$);
			\begin{scope}
				\draw (0,0) node (l1a) {} .. controls +(0.2,0) and (0.3,0.2) .. (0.5,0.2) node (l1b) {};
			\end{scope}
			\begin{scope}[yshift=10]
				\draw (0.5,0.2) node (l1c) {} .. controls +(0.3,0) .. (1,0.3) node (l1d) {};
			\end{scope}
			\begin{scope}[yshift=15]
				\draw (1,0.3) node (l1e) {} .. controls +(0,0.3) .. (1.5,0.6) node (l1f) {};
			\end{scope}
			\begin{scope}[yshift=30]
				\draw (1.5,0.6) node (l1g) {} -- (2,0.8) node (l1h) {};
			\end{scope}
			\begin{scope}[every path/.style={lightgray, dotted}]
				\draw (l1b) -- (l1c);
				\draw (l1d) -- (l1e);
				\draw (l1f) -- (l1g);
			\end{scope}
		\end{scope}
		\begin{scope}[xshift=40, yshift=-110, scale=1.5]
			\draw[->] (0,0) -- node[left] {$\wh \ell$} (0,2);
			\draw (0,0) node (a2a) {} -- (2,0) node (a2b) {};
			\draw[decoration={calligraphic brace,amplitude=5pt}, decorate, line width=1.25pt]
				($(a2b) + (0,-0.1)$) -- node[yshift=-12] {$C$} ($(a2a) + (0,-0.1)$);
			\begin{scope}[yslant=0.35]
				\draw (0,0) node (l1a) {} .. controls +(0.2,0) and (0.3,0.2) .. (0.5,0.2) node (l1b) {};
				\draw (0.5,0.2) node (l1c) {} .. controls +(0.3,0) .. (1,0.3) node (l1d) {};
				\draw (1,0.3) node (l1e) {} .. controls +(0,0.3) .. (1.5,0.6) node (l1f) {};
				\draw (1.5,0.6) node (l1g) {} -- (2,0.8) node (l1g) {};
			\end{scope}
		\end{scope}
		\begin{scope}[xshift=0, yshift=-110, scale=1.5]
			\draw[->] (-0.4,0.8) -- (0.4,0.8);
		\end{scope}
		% \begin{scope}[every path/.style={dashed}]
		% 	\draw (Ca) -- (a1a);
		% 	\draw (Cb) -- (a1b);
		% 	\draw (Ca) -- (a2a);
		% 	\draw (Cb) -- (a2b);
		% \end{scope}
	\end{tikzpicture}
\end{equation*}

\begin{equation*}
	\index{path!partly rounded}
	\index{array!two-dimensional}
	\index{foreach}
	\index{variable!pgfmathsetmacro}
	\index{arrows!bar}
	\begin{tikzpicture}
		\def\twigs{{{0.0,0.1},{0.1,0.1},{0.3,0.15},{0.4,0.2},{0.5,0.3},{0.6,0.3},{0.7,0.4},{0.8,0.45}, {1.0,0.55},{1.2,0.6},{1.3,0.6},{1.5,0.7},{1.6,0.7},{1.7,0.75},{1.8,0.8},{1.9,0.8},{2.0,0.9}}}
		% \def\twigs
		\pgfmathsetmacro{\twigmax}{16}
		\coordinate (p) at (0,0);
		\begin{scope}[xscale=2, yscale=1, xshift=-30, yshift=100]
			\filldraw[lightgray] (p) {[rounded corners=15] -- (0.8,0.5) -- (2.3,0.5)} -- cycle;
			\foreach \i in {0,...,\twigmax}
			{
				\draw (\twigs[\i][0],0) node[point] (a\i) {} -- (\twigs[\i][0],\twigs[\i][1]);
			}
			\draw[dashed] (-0.2,0.5) -- (2.2,0.5);
			\draw[Bar-Bar] (2.4,0) --node[right] {$2\ep$} (2.4,0.5) ;
			\draw[brace]
				(2,-0.2) -- node[yshift=-10] {$A^*$} (1,-0.2);
			\draw[brace]
				(2,-0.8) -- node[yshift=-10] {$A$} (0,-0.8);
		\end{scope}
		\node[point] at (p) {};
		\draw[dotted] (p) edge (-4,4) edge (4,4);
		\node at (0.4,1.5) {$Y^*$};
	\end{tikzpicture}
\end{equation*}

\begin{equation*}
	\index{arrows!bar}
	\begin{tikzpicture}
		\begin{scope}[every node/.style={point}]
			\node (p) at (0,0) [label=below:$p$] {};
			\node (m) at (0,2) {};
			\node (x) at (-2,4) [label=right:$x$] {};
			\node (y) at (2,4) [label=left:$y$] {};
		\end{scope}
		\node (ml) at ($(m) + (1.5,-0.3)$) {$x \wedge y$};
		\draw[->] (ml)  -- (m);
		\draw (p) -- (m) -- (x);
		\draw (m) -- (y);
		\begin{scope}[every path/.style={blue,{Bar-Bar}}]
			\draw ($(p)+(-0.6,0)$) -- node[left] {$\ell(x \wedge y)$} ($(m)+(-0.6,0)$);
			\draw ($(p)+(-0.2,0)$) -- ($(m)+(-0.185,-0.077)$) -- node[below left] {$\ell(x)$} ($(x)+(-0.141,-0.141)$);
			\draw ($(p)+(0.2,0)$) -- ($(m)+(0.185,-0.077)$) -- node[below right] {$\ell(y)$} ($(y)+(0.141,-0.141)$);
			% \draw ($(p)+(-0.2,0)$) -- ($(m)+(-0.2,0)$) -- node[right left] {$\ell(x)$} ($(x)+(-0.141,-0.141)$);
		\end{scope}
	\end{tikzpicture}
\end{equation*}

\begin{equation*}
	\index{variable!tikzmath}
	\index{foreach!tikzmath}
	\index{math library}
	\index{library!math|see {math library}}
	\index{every!node}
	\index{every!path}
	\begin{tikzpicture}[xscale=2, yscale=1.5]
		\begin{scope}[every node/.style={point},every path/.style={thick}]
			\node (o0) at (0,0) {};
			\node (n0) at (-1,1) {};
			\node (n000) at (-1.833,1.833) {};
			\node (n01) at (-0.5,1.5) {};
			\node (n010) at (-0.833,1.833) {};
			\node (n011) at (-0.167,1.833) {};
			\draw[blue] (o0) -- (n0) -- (n000);
			\draw[red] (n0) -- (n01) -- (n011);
			\draw[blue] (n01) -- (n010);
			\begin{scope}[xshift=40]
				\tikzmath{
					coordinate \p;
					\omegapoints = 10;
					\p = (1,1);
					\colour = 0;
					\t = 1;
					int \n; % Need this otherwise nodes get named like o1.0, while '\draw (o\n)' evaluates like '\draw (o1)', for some stupid reason.
					for \n in {1,...,\omegapoints}{
						{
							\node (o\n) at (\p) {};
						};
						int \prev;
						\prev = \n - 1;
						if \colour==0 then {
							\colour = 1;
							\p = (\p) + ({-1/(\n+1)},{1/(\n+1)});
							{
								\draw[red] (o\prev) -- (o\n);
							};
						} else {
							\colour = 0;
							\p = (\p) + ({1/(\n+1)},{1/(\n+1)});
							{
								\draw[blue] (o\prev) -- (o\n);
							};
						};
					};
				}
				\coordinate (top) at (0.693,3.1);
			\end{scope}
			\node (ninf) at ($(top) + (0,0.4)$) {};
			\node (ninf0) at ($(top) + (-0.4,0.8)$) {};
			\node (ninf1) at ($(top) + (0.4,0.8)$) {};
			\draw[blue] (ninf) -- (ninf0);
			\draw[red] (ninf) -- (ninf1);
		\end{scope}
		\node at ($(top) + (0,0.2)$) {$\vdots$};
		\node [below=1pt of o0] {$()$};
		\node [left=1pt of n0] {$((0,1))$};
		\node [left=1pt of n000] {$((0,3))$};
		\node [right=1pt of n01] {$((0,1),(1,1))$};
		\node [right=1pt of n011] {$((0,1),(1,2))$};
		\node [above=1pt of n010] {$((0,1),(1,1),(0,1))$};
		% \node at ($(n010) + (-0.5,1)$) {$((0,1),(1,1),(0,1))$};
		\node [right=1pt of o1] {$((1,1))$};
		\node [left=1pt of o2] {$((1,1),(0,1))$};
		\node [right=1pt of ninf] {$((1,1), (0,1), \ldots)$};
		\node [left=1pt of ninf0] {$((1,1), (0,1), \ldots,(1,1))$};
		\node [above=1pt of ninf1] {$((1,1), (0,1), \ldots,(0,1))$};
	\end{tikzpicture}
\end{equation*}

\begin{equation*}
	\index{node shaped as coordinate}
	\index{decorations library!coloured}
	\begin{tikzpicture}
		\coordinate	(p) at (0,0);
		\coordinate	(b) at (1,3);
		\coordinate	(t) at (2,6);
		\draw[blue] (p) --
			node[shape=coordinate,pos=0.1] (a0) {}
			node[shape=coordinate,pos=0.3] (a1) {}
			node[shape=coordinate,pos=0.6] (a2) {}
			node[shape=coordinate,pos=0.7] (a3) {}
			node[shape=coordinate,pos=0.9] (a4) {}
			node[shape=coordinate,pos=1.0] (a5) {}
			(b);
		\begin{scope}
			\clip (-2,6) rectangle (1,0);
			\foreach \n in {0,...,5}
			{
				\draw[gray] (a\n) -- ($(a\n)+(-2,6)$);
			}
		\end{scope}
		\draw[blue] (b) -- (t);
		\node at (2,6.5) {$B_\alpha$};

		\draw[->] (3,3) -- node[midway, above] {$r_\alpha$} (5,3);

		\draw[{Bar-Bar}] (7,0) -- (7,6);
		\node at (7,6.5) {$\Q \cap [0,1]$};

		\path (7,0) -- 
			node[point,pos=0.2] (x0) {}
			node[point,pos=0.25] {}
			node[point,pos=0.4] {}
			node[point,pos=0.55] {}
			node[point,pos=0.6] {}
			node[point,pos=0.65] {}
			node[point,pos=0.8] {}
			node[point,pos=0.9] (x1) {}
			(7,6);

		\draw[brace] ($(x1) + (0.5,0)$) -- node[midway, xshift=35] {$\{q_0, \ldots, q_n\}$} ($(x0) + (0.5,0)$);

		\draw[thick, red] (b) -- (t);
		\draw[brace, pen colour={red}] ($(t)+(0.5,0)$) -- node[midway, xshift=40, yshift=-4, red] {$B_\alpha \setminus \bigcup_{\beta<\alpha}B_\beta$} ($(b)+(0.5,0)$);

		\path (b) -- 
			node[point,pos=0.1] {}
			node[point,pos=0.2] {}
			node[point,pos=0.3] {}
			node[point,pos=0.6] {}
			node[point,pos=0.8] {}
			(t);
	\end{tikzpicture}
\end{equation*}

\begin{equation*}
	\index{array!listofitems}
	\index{package!pgfplots|see pgfplots}
	\index{package!listofitems|see array!listofitems}
	\index{foreach!tikzmath}
	\index{variable!tikzmath}
	\index{math library}
	\index{pgfplots}
	\index{pgfplots!colormap}
	\index{if statement!tikzmath}
	\begin{tikzpicture}
		\setsepchar{:/;/,}
		\readlist\branches{0,red;20,blue;15,red;30,blue;10,blue:0,blue;15,red;20,blue;15,red;15,red}
		\edef\brancheslen{\listlen\branches[1]}
		\pgfplotsset{colormap={redblue}{color=(blue) color=(red)}}
		\newcommand{\drawbranch}[6]{ % x, y, direction, size, colour, nodename
			\begin{scope}[xshift=#1,yshift=#2,xscale=#3]
				\begin{axis}[
					hide axis,
					point meta rel=per plot,
					width=#4,
					height=#4,
					colormap name=redblue,
					clip=false,
					xmin=0,
					ymin=0,
					anchor=south west
				]
					\addplot[
						mesh,
						mark=none,
						point meta={sin(1000/x))>0 ? 1 : 0},
						colormap access=piecewise constant,
						samples=200,
						domain=0.00001:5,
						ultra thick
					] 
						{x};
					\node[point,minimum size=4,#5] at (0,0) (#6) {};
				\end{axis}
			\end{scope}
		}
		\drawbranch{0}{0}{-1}{100}{blue}{c0}
		\drawbranch{-30}{30}{1}{198}{blue}{c1}
		\drawbranch{0}{60}{-1}{165}{red}{c2}
		% \drawbranch{10}{70}{1}{100}{blue}{c3}
		\tikzmath{
			real \x, \y;
			int \dir;
			int \branch;
			for \branch in {1,2}{
				if \branch==1 then {
					\x = -60;
					\dir = -1;
				} else {
					\x = 60;
					\dir = 1;
				};
				\y = 120;
				int \i;
				for \i in {1,...,\brancheslen}{
					\x = \x + \branches[\branch,\i,1]*\dir;
					\y = \y + \branches[\branch,\i,1];
					\dir = 0 - \dir;
					let \col = \branches[\branch,\i,2];
					{\drawbranch{\x}{\y}{\dir}{80}{\col}{b{\branch}n{\i}}};
				};
			};
		}
		\node[xshift=-35,yshift=230] {$\vdots$};
		\node[xshift=65,yshift=220] {$\vdots$};
	\end{tikzpicture}
\end{equation*}