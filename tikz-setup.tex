%!TEX root = tikz-figures.tex

\usetikzlibrary{
	external, % Externalise tikzfigures to speed up compile. Needs %!TEX options = -shell-escape
	cd,
	positioning,
	arrows,
	graphs,
	arrows.meta,
	calc,
  math,
	patterns,
	patterns.meta,
	decorations.pathreplacing,
  decorations.markings,
	calligraphy
}

% PGF backwards compatability mode
\pgfplotsset{compat=1.16}

% \usegdlibrary{trees}

% Set up externalisation
\tikzexternalize[prefix=figures/]

% Turn off externalisation for Tikz cd figures (because it doesn't work)
% https://tex.stackexchange.com/a/172432
\AtBeginEnvironment{tikzcd}{\tikzexternaldisable}
\AtEndEnvironment{tikzcd}{\tikzexternalenable}

% Tikz styles
\tikzset{
	world/.style={draw,circle,outer sep=0pt,inner sep=0,minimum size=15},
	point/.style={draw=black,fill=black,opacity=1,circle,outer sep=0pt,inner sep=0,minimum size=2},
	facefill/.style={fill=blue!20,fill opacity=0.4}
}
\tikzgraphsset{
	poset/.style={grow up=#1, branch right=#1, empty nodes, simple, nodes=point},
	poset/.default=1cm
}
\tikzset{negated/.style={
        decoration={markings,
            mark= at position 0.5 with {
                \node[transform shape] (tempnode) {$\backslash$};
            }
        },
        postaction={decorate}
    }
}
\tikzset{
  brace/.style={decoration={calligraphic brace,amplitude=5pt}, decorate, line width=1.25pt}
}
\tikzset{
  ->-/.style={
    decoration={
      markings,
      mark=at position .5 with {\arrow{>}}
    },
    postaction={decorate}
  }
}


% Distance between points
\makeatletter
\newcommand{\Distance}[3]{% % from https://tex.stackexchange.com/q/56353/121799
\tikz@scan@one@point\pgfutil@firstofone($#1-#2$)\relax  
\pgfmathsetmacro{#3}{round(0.99626*veclen(\the\pgf@x,\the\pgf@y)/0.0283465)/1000}
}% Explanation: the calc library allows us, among other things, to add and
% subtract points, so ($#1-#2$) is simply the difference between the points
% #1 and #2. The combination \tikz@scan@one@point\pgfutil@firstofone extracts
% the coordinates of the new point and stores them in \pgf@x and \pgf@y. 
% They get fed in veclen, and \pgfmathsetmacro stores the result in #3. 
% EDIT: included fudge factor, see https://tex.stackexchange.com/a/22702/121799
\makeatother

% Barycentre of points
% Modified from the tikz source tikz.code.tex
\makeatletter
\tikzdeclarecoordinatesystem{bc}
{%
  {%
    \pgf@xa=0pt% point
    \pgf@ya=0pt%
    \pgf@xb=0pt% sum
    \tikz@bc@dolist#1,,%
    \pgfmathparse{1/\the\pgf@xb}%
    \global\pgf@x=\pgfmathresult\pgf@xa%
    \global\pgf@y=\pgfmathresult\pgf@ya%
  }%
}%

\def\tikz@bc@dolist#1,{%
  \def\tikz@temp{#1}%
  \ifx\tikz@temp\pgfutil@empty%
  \else
    \pgf@process{\pgfpointanchor{#1}{center}}%
    \pgfmathparse{1}%
    \advance\pgf@xa by\pgfmathresult\pgf@x%
    \advance\pgf@ya by\pgfmathresult\pgf@y%
    \advance\pgf@xb by\pgfmathresult pt%
    \expandafter\tikz@bc@dolist%
  \fi%
}
\makeatother